\documentclass{article}
\usepackage{geometry}
\geometry{margin=0.5in}
\usepackage{multicol}
\setlength{\columnsep}{1cm}
\thispagestyle{empty}

\usepackage{paralist}

\begin{document}
\begin{center}
  \begin{tabular}{r l}
    {\huge\textbf{Andrew}}             & {\huge\textbf{Farabow}}        \\
    \hspace{35pt} github.com/andrewaf1 & linkedin.com/in/andrew-farabow \\
    703-474-6270                       & contact@andrewfarabow.com      \\
  \end{tabular}


  \begin{flushleft}
    \begin{multicols}{2}
      {\large\textbf{\underline{\smash{Education}}}} \\
      \textbf{Virginia Tech (expected grad 2023)} \\
      GPA: 3.24 \; B.S. in Computer Science w/ Stat minor \\
      Elective Courses: Restricted Research, Mathematical Statistics 1-2 (Probability and Inference), Intro to Data Analytics and Visualization\\
      \textbf{Gonzaga College High School	(2015 - 2019)} \\
      GPA 3.98 \\


      \columnbreak
      {\large\textbf{\underline{\smash{Skills}}}} \\
      {\textbf{Programming:}} Python, C, Java, R, Matlab \\
      {\textbf{Frameworks:}} PyTorch, Scikit-learn, Keras, Numpy, OpenCV, Pandas, Matplotlib, RLLib, OpenAI Gym \\
      {\textbf{Other:}} deep learning, recurrent and convolutional neural networks, reinforcement learning, GANs, autoencoders, data analytics, statistical learning, Linux, Git, Kubernetes, LaTeX, Agile \\

    \end{multicols}

    {\large\textbf{\underline{\smash{Work Experience}}}} \\

    \textbf{Research Assistant - Sanghani Center (Virginia Tech) \hfill May. 2021 - present}
    \begin{compactitem}
      \itemsep0em
      \item Spearheading the effort to create an open-source library of epidemiological models for forecasting the COVID-19 pandemic and the seasonal flu, under the direction of Prof. Naren Ramakrishnan.
      \item Utilized object-oriented design to maximize code reuse and simplify the creation of new models and datasets, wrote tests in PyTest, and created online documentation.
      \item Implemented compartmental, classical time-series, and machine learning models, as well as a range of datasets and evaluation metrics.
    \end{compactitem}

    \textbf{Research Assistant - BIST (Virginia Tech) \hfill Nov. 2019 - present (school year)}
    \begin{compactitem}
      \itemsep0em
      \item Working on a Center for Bioinspired Science and Technology project, led by Prof. Rolf Mueller, involving the use of bat-inspired biomimetic sonar and deep learning for robotic navigation in forested environments.
      %\item Previously used the DeepLabCut software package to track bats' movements in images. 
      \item Helped develop a ConvNet-based algorithm to identify the position of the sonar sensor within a forest area
    \end{compactitem}

    \textbf{Research Assistant - Hume Center (Virginia Tech) \hfill Sept. 2019 - present (school year)}
    \begin{compactitem}
      \itemsep0em
      \item Applying reinforcement learning algorithms to simulations designed to mimic defense systems as part of a Raytheon-Virginia Tech partnership.
      \item Previously designed and trained object-detecting convolutional neural network architectures, which achieved 97\% accuracy on the classification phase of the Lockheed Martin AlphaPilot Dataset and were deployed to a drone's computer to aid in navigation.
    \end{compactitem}

    \textbf{Machine Learning Engineer Intern - Decipher Technology Studios \hfill 2018 - 2020 (summers)}
    \begin{compactitem}
      \itemsep0em
      \item Worked on a small team to develop Sense, a product which uses deep reinforcement learning to control the resources allocated to a microservice, striking a balance between server performance and hosting cost (predictive autoscaling).
      \item Implemented a library of policy gradient, Q-Learning, and actor-critic deep reinforcement learning algorithms (DQN, DDPG, A2C, PPO, SAC, etc) in PyTorch.
      \item Wrote a microservice environment simulator for offline training and built a Sense microservice for online training and deployment (on Openshift and Elastic Kubernetes Service).
      \item Added recurrent and convolutional layers to the neural networks to better leverage autocorrelation within the data.
      %\item Collected metrics using Prometheus and Gatling and tested various model architectures on the data
      \item Contributed to a related project called the Sense Log Anomaly Detector (LAD), which uses a recurrent autoencoder to identify anomalous log lines. Worked on improving the autoencoder by adding a self-attention mechanism.
    \end{compactitem}


    {\large\textbf{\underline{\smash{Activities}}}} \\

    \textbf{Judging Coordinator - VTHacks Organizing Team \hfill 2019 - present}
    \begin{compactitem}
      \itemsep0em
      %\item Reached out to potential corporate sponsors and faculty judges for Virginia Tech's hackathon
      %\item Handled judging logistics during the event and took note of improvements to implement next year
      \item Responsible for recruiting faculty judges and managing judging logistics during the event.
    \end{compactitem}

    \textbf{Stage Manager - Gonzaga Dramatic Association Stage Crew \hfill 2017 - 2019}
    \begin{compactitem}
      \itemsep0em
      %\item Led a team of over 20 students in the construction of a structure over 20 ft. wide and 8 ft. tall
      %\item Called cues during shows, maintained safe working conditions and quickly diagnosed and fixed technical issues in a high-pressure environment
      \item Led a team of over 20 students in the construction of a structure over 20 ft. wide and 8 ft. tall.
      \item Quickly diagnosed and fixed technical issues in a high-pressure environment.
    \end{compactitem}

    \textbf{Participant and Mentor - HackBI  (Bishop Ireton High School Hackathon) \hfill 2017 \& 2018}
    \begin{compactitem}
      \itemsep0em
      \item Won best overall in a programming contest by writing an app that makes use of machine learning and computer vision techniques to interpret hand-written text.
      \item Returned to HackBI in 2018 to mentor teams and teach deep learning concepts.
    \end{compactitem}


    {\large\textbf{\underline{\smash{Projects}}}} \\
    \textbf{Computable AI} - co-author of a blog on machine learning, writing a Fundamentals of Deep RL series. \\
    \textbf{Machine Learning Templates} - flexible PyTorch implementations of a supervised learning neural network, autoencoder, GAN, and evolutionary algorithm designed for future machine learning projects. \\
    \textbf{Grease Lights and Magic Mirror} - coded and designed circuits for custom Arduino and Raspberry Pi-based lighting effects and optical illusions featured in high school theater productions.

  \end{flushleft}
\end{center}


\end{document}
