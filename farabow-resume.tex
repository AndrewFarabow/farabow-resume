\documentclass{article}
\usepackage{geometry}
\geometry{margin=0.5in}
\usepackage{multicol}
\setlength{\columnsep}{1cm}
\thispagestyle{empty}

\usepackage{paralist}

\begin{document}
\begin{center}
  \begin{tabular}{r l}
    {\huge\textbf{Andrew}}             & {\huge\textbf{Farabow}}        \\
    \hspace{35pt} github.com/andrewaf1 & linkedin.com/in/andrew-farabow \\
    703-474-6270                       & contact@andrewfarabow.com      \\
  \end{tabular}


  \begin{flushleft}
    \begin{multicols}{2}
      {\large\textbf{\underline{\smash{Education}}}} \\
      \textbf{Virginia Tech (expected grad 2023)} \\
      GPA: 3.24 \; B.S. in Computer Science w/ Stat minor \\
      Elective Courses: Restricted Research, Mathematical Statistics 1-2 (Probability and Inference), Intro to Data Analytics and Visualization\\
      \textbf{Gonzaga College High School	(2015 - 2019)} \\
      GPA 3.98 \\


      \columnbreak
      {\large\textbf{\underline{\smash{Skills}}}} \\
      {\textbf{Programming:}} Python, C, Java, R, Matlab \\
      {\textbf{Frameworks:}} PyTorch, Scikit-learn, Keras, Numpy, OpenCV, Pandas, Matplotlib, RLLib, OpenAI Gym \\
      {\textbf{Other:}} deep learning, recurrent and convolutional neural networks, reinforcement learning, GANs, autoencoders, data analytics, statistical learning, Linux, Git, Kubernetes, LaTeX, Agile \\

    \end{multicols}

    {\large\textbf{\underline{\smash{Work Experience}}}} \\

    \textbf{Research Assistant - Sanghani Center (Virginia Tech) \hfill May. 2021 - present}
    \begin{compactitem}
      \itemsep0em
      \item Developing a recurrent nueral network (RNN) model to forecast influenza cases for the CDC FluSight Competition.
      \item Spearheading the effort to create an open-source library of epidemiological models for forecasting the COVID-19 pandemic and the seasonal flu, under the direction of Prof. Naren Ramakrishnan (not yet released).
      \item Created a user-friendly, scikit-learn inspired interface and structured the library to maximize code reuse and simplify the creation of new models and datasets.
      \item Implemented compartmental, classical time-series, and machine learning models, as well as a range of datasets and evaluation metrics.
      %\item Wrote tests in PyTest and created online documentation.
    \end{compactitem}

    \textbf{Research Assistant - BIST (Virginia Tech) \hfill Nov. 2019 - present (school year)}
    \begin{compactitem}
      \itemsep0em
      \item Working on a Center for Bioinspired Science and Technology project, led by Prof. Rolf Mueller, involving the use of bat-inspired biomimetic sonar and deep learning for robotic navigation in forested environments.
      %\item Previously used the DeepLabCut software package to track bats' movements in images. 
      \item Helped develop a ConvNet-based algorithm to predict the position of the sonar sensor within a forest area.
    \end{compactitem}

    \textbf{Research Assistant - Hume Center (Virginia Tech) \hfill Sept. 2019 - Dec. 2021 (school year)}
    \begin{compactitem}
      \itemsep0em
      \item Built a grid-based, OpenAI Gym-compatible simulation called SensorGrid that replicated key aspects of drone sensing and navigation challenges in a simplified environment, useful for testing reinforcement learning models before deployment to a more computationally-expensive environment, as part of the Raytheon RAAIDS project.
      \item Designed and trained a Resnet-based object-detecting convolutional neural network architecture, which achieved 97\% accuracy on the classification phase of the Lockheed Martin AlphaPilot Dataset and was deployed to a drone's computer to aid in navigation.
      \item Participated in the IC CAE Scholars Program, which involves conducting research with the Hume Center and participating in a number of events (seminars, workshops, etc).
    \end{compactitem}

    \textbf{Machine Learning Engineer Intern - Decipher Technology Studios \hfill 2018 - 2020 (summers)}
    \begin{compactitem}
      \itemsep0em
      \item Improved performance of a recurrent autoencoder used to identify anomalies in service logs by adding self-attention.
      \item Worked on a small team to develop a predictive autoscaler that uses deep reinforcement learning to control the resources allocated to a microservice, striking a balance between performance and hosting cost.
      \item Wrote PyTorch implementations of policy gradient, Q-Learning, and actor-critic deep reinforcement learning algorithms (DQN, DDPG, A2C, PPO, SAC, etc).
      \item Wrote a simulator for offline training and a microservice for online training and deployment (on Openshift and EKS).
      \item Added recurrent and convolutional layers to the neural networks to better leverage autocorrelation within the data.
      %\item Collected metrics using Prometheus and Gatling and tested various model architectures on the data
    \end{compactitem}


    {\large\textbf{\underline{\smash{Activities}}}} \\

    \textbf{Head of Logistics - VTHacks Organizing Team \hfill 2019 - present}
    \begin{compactitem}
      \itemsep0em
      %\item Reached out to potential corporate sponsors and faculty judges for Virginia Tech's hackathon
      %\item Handled judging logistics during the event and took note of improvements to implement next year
      \item Overseeing the team responsible for managing the budget, purchasing meals, recruiting faculty judges, and other tasks.
    \end{compactitem}

    \textbf{Stage Manager - Gonzaga Dramatic Association Stage Crew \hfill 2017 - 2019}
    \begin{compactitem}
      \itemsep0em
      %\item Led a team of over 20 students in the construction of a structure over 20 ft. wide and 8 ft. tall
      %\item Called cues during shows, maintained safe working conditions and quickly diagnosed and fixed technical issues in a high-pressure environment
      \item Led a team of over 20 students in the construction of a structure over 20 ft. wide and 8 ft. tall.
      \item Quickly diagnosed and fixed technical issues in a high-pressure environment.
    \end{compactitem}

    \textbf{Participant and Mentor - HackBI  (Bishop Ireton High School Hackathon) \hfill 2017 \& 2018}
    \begin{compactitem}
      \itemsep0em
      \item Won best overall in a programming contest by writing an app that makes use of machine learning and computer vision techniques to interpret hand-written text.
      \item Returned to HackBI in 2018 to mentor teams and teach deep learning concepts.
    \end{compactitem}


    {\large\textbf{\underline{\smash{Projects}}}} \\
    \textbf{Computable AI} - co-author of a blog on machine learning, writing a Fundamentals of Deep RL series. \\
    \textbf{Machine Learning Templates} - flexible PyTorch implementations of a supervised learning neural network, autoencoder, GAN, and evolutionary algorithm designed for future machine learning projects. \\
    \textbf{Grease Lights and Magic Mirror} - coded and designed circuits for custom Arduino and Raspberry Pi-based lighting effects and optical illusions featured in high school theater productions.

  \end{flushleft}
\end{center}


\end{document}
