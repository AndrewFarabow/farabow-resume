\documentclass{article}
\usepackage{geometry}
\geometry{margin=1in}
\usepackage{multicol}
\setlength{\columnsep}{1cm}

\begin{document}
  \begin{center}
    \begin{tabular}{r l}
      {\huge\textbf{Andrew}} & {\huge\textbf{Farabow}} \\
      \hspace{35pt} github.com/andrewaf1 & linkedin.com/in/andrew-farabow \\
      703-474-6270 & aafarabow@gmail.com \\
    \end{tabular}


  \begin{flushleft}
    \begin{multicols}{2}
      {\large\textbf{\underline{\smash{Education}}}} \\
       \textbf{Virginia Tech	(2019 - present)} \\
      Major: General Engineering (Computer Science) \\
      \textbf{Gonzaga College High School	(2015 - 2019)} \\
      GPA 3.98 \\
     

    \columnbreak
    {\large\textbf{\underline{\smash{Skills}}}} \\
    {\textbf{Programming:}} Python, Java \\
    {\textbf{Frameworks:}} Numpy, PyTorch, OpenAI Gym, OpenCV, Pandas \\
    {\textbf{Other:}} neural networks, reinforcement learning, GANs, data analytics, Git, LaTeX, Gatling \\

    \end{multicols}

    {\large\textbf{\underline{\smash{Work Experience}}}} \\
    \textbf{Undergraduate Research Assistant - Hume Center \hfill 2019 - present}
    \begin{itemize}
      \item Working for Prof. Daniel Doyle on SPIDER (Space Interceptor/Detector/Evaluator/Revitalizer)
    \end{itemize}


    \textbf{Machine Learning Engineer Intern - Decipher Technology Studios \hfill 2018 - present}
    \begin{itemize}
      \item Working on a small team to develop a new product which provides deep reinforcement learning-powered predictive autoscaling for Decipher’s Grey Matter service mesh
      \item Studied and implemented policy gradient, Q-Learning, and actor-critic approaches to deep reinforcement learning (DQN, DDPG, A2C, PPO, SAC, etc)
      \item Wrote a microservice environment simulator for offline training with another intern and created a rule-based autoscaler to jumpstart training via imitation learning.
      \item Configured and deployed demos of Sense to AWS and Openshift for client meetings and major conferences.
      \item Added Gated Recurrent Units and Convolutional Layers to the neural network to better leverage time-series data
      \item Collected metrics using Prometheus and Gatling and tested various model architectures on the data using supervised learning
      \item Compared the performance of different configurations of Sense and kept detailed records of the results
    \end{itemize}


    {\large\textbf{\underline{\smash{Activities}}}} \\
    \textbf{Gonzaga Dramatic Association Stage Crew \hfill 2017 - 2019}
    \begin{itemize}
      \item Led a 20-member team for two productions as stage manager (2018-2019)
      \item Designed and coordinated the construction of a structure over 20 ft. wide and 8 ft. tall
      \item Called cues during shows, maintained safe working conditions and solved problems in a high-pressure environment
      \item Worked with the stage manager to quickly diagnose and fix technical issues as assistant stage manager (2017-2018) before being promoted
    \end{itemize}

    \textbf{HackBI  (Bishop Ireton High School Hackathon)}

    \begin{itemize}
      \item Won best overall in a programming contest by writing an app that makes use of machine learning and computer vision techniques to interpret hand-written text
      \item Returned to HackBI in 2018 to mentor teams and teach deep learning concepts
    \end{itemize}

    {\large\textbf{\underline{\smash{Projects}}}} \\
    \textbf{Computable AI} - co-author of a blog on machine learning, writing a Fundamentals of Deep RL series \\
    \textbf{Machine Learning Templates} - flexible PyTorch implementations of a supervised learning neural network, autoencoder, GAN, and evolutionary algorithm designed for future machine learning projects \\
    \textbf{Grease Lights and Magic Mirror} - coded and designed circuits for custom Arduino and Raspberry Pi-based lighting effects and optical illusions featured in high school theater productions


  \end{flushleft}
  \end{center}


\end{document}
